\documentclass{resume} % Use the custom resume.cls style
\usepackage[left=0.75in,top=0.6in,right=0.75in,bottom=0.6in]{geometry} % Document margins
\usepackage{multicol,setspace}
\newcommand{\tab}[1]{\hspace{.2667\textwidth}\rlap{#1}}
\newcommand{\itab}[1]{\hspace{0em}\rlap{#1}}
\name{Zhongyuan (Leo) Liu} % Your name
\address{Room 313, West Wing, Level 3 of UNSW Business School, Kensington 2052}
\address{(+61)421078664 \\ auleo.liu@gmail.com} % Your phone number and email

\begin{document}
\setstretch{0.8}

\begin{rSection}{Research Field of Interests}
	Innovation, Entrepreneurship and Organization

\end{rSection}
\begin{rSection}{Education}

	{\bf University of New South Wales} \hfill {\em 2017 - 2022}
	\\ PhD in Finance
	\\ School of Banking and Finance\\
	\\{\bf University of New South Wales} \hfill {\em 2015 - 2017}
	\\ Master of Philosophy, Finance.
	\\ School of Banking and Finance
	\\
	\\{\bf Bond University} \hfill {\em 2009 - 2012}
	\\ Bachelor of Commerce
\end{rSection}

\begin{rSection}{Working Papers}
	{\bf Advanced Manufacturing and Process Innovation, Productivity and Growth} (with Elvira Sojli and Wing Wah Tham) (Job Market Paper)\\
	{\bf Abstract:}\\
	We propose a new time-varying measure of advanced manufacturing and process innovations based on patent invention titles from 1926 to 2019 across 51 countries. Using this measure, we provide evidence that these innovations are important for firm and aggregate economic growth through technological spillover to product innovations and increased profitability. We find that patents with advanced processes have higher forward citations, number of claims, and economic value than product patents of the same technology class and cohort. Firms with higher quality process innovation are associated with higher profits, sales, capital, employment, and total factor productivity in the short run, while product innovation plays a larger role in the longer run. At the aggregate level, both advanced manufacturing and product innovations are positively related to U.S. economic growth. In cross-country analysis, we find that product innovation is more important than process innovation for emerging countries with low labor costs for growth convergence towards the U.S. economy, but the opposite holds for productivity.
		{\bf Conferences:}
	\begin{itemize}
    \item AIEA-NBER Conference on Innovation and Entrepreneurship 2022
		\item ABFER (Asian Bureau of Finance and Economic Research) 2022
		\item CEPR (Centre for Economic Policy Reserch) Rising Asia 2022
		\item EARIE (European Association for Research in Industrial Economics) 2022
		\item SEM (Society for Economic Measurement) 2022
		\item FMCG (Financial Market and Corporate Governance) 2022
		\item Asian Innovation Economics Conference 2021 (invitation-only)
		\item Business Financing and Banking Research Group Annual Workshop 2021 (invitation-only)
		\item Macroeconomics Reading Group Workshop 2021 (invitation-only)
	\end{itemize}

	{\bf Common Ownership and Antitrust Enforcement: Evidence from the Court} (with Huaizhou Li, Ronald Masulis and Jason Zein)\\
	{\bf Abstract:}\\
	An increasing number of publicly-owned rival firms are jointly held by a small group of large institutional investors. This has led to serious concerns regarding the economic implications of such common ownership. We analyze U.S. legal cases to uncover links between common ownership and antitrust litigation. This approach allows us to establish a direct connection between common ownership and anti-competitive product market behavior. We empirically show that there is a significant positive relationship between common ownership and the probability that a pair of firms face the same antitrust prosecution. We use mergers between financial institutions as exogenous variations in common ownership to provide causal evidence on this proposed relation. We also show that firm-pairs subject to antitrust litigation are characterized by more socially connected directors, pointing to a possible coordination mechanism.

	Conferences:

	\begin{itemize}
		\item CELS 2022 (Conferences on Empirical Legal Studies)
		\item FMCG 2022
	\end{itemize}

	{\bf Re-defining Industry Classification} (with Roni Michaely, Elvira Sojli and Wing Wah Tham)\\

	% {\bf Product Variety of the Firms} (with Wing Wah Tham and Elvira Sojli)\\
	% This paper proposes a new proxy for measuring the real time number and value of key goods and serviced produced by firms. We use the new measure in three economic applications: we study the role of stock market prices as information aggregates; we investigate the relation between R\&D as a barrier to entry and product innovation; and we examine the role of product variety and quality on economic growth. Our empirical applications show that: stock market reaction around trademark commercializations align with future product quality; firms R\&D expenses that arise endogenously spur more product variety; and the majority of economic growth derives from product variety and quality rather than creative destruction.
	{\bf Reassessing the Measures of Carbon Emission Intensity} (with Harrison Hong, Elvira Sojli and Wing Wah Tham)\\

	{\bf Clean Innovation} (with Elvira Sojli and Wing Wah Tham)\\
	% {\bf Clean Innvoation} (with Wing Wah Tham and Elvira Sojli)\\
	% This paper proposes a new proxy for measuring the real time number and value of key goods and serviced produced by firms. We use the new measure in three economic applications: we study the role of stock market prices as information aggregates; we investigate the relation between R\&D as a barrier to entry and product innovation; and we examine the role of product variety and quality on economic growth. Our empirical applications show that: stock market reaction around trademark commercializations align with future product quality; firms R\&D expenses that arise endogenously spur more product variety; and the majority of economic growth derives from product variety and quality rather than creative destruction.

	{\bf Measuring CEO's General Ability by Recovering Boards' Perception} (with Lixiong Guo) \\
	\\ Conferences: FMA 2017 semi-finalist for best paper award, CAFM 2017, AFBC 2017.

		{\bf Easy to Parse, Easy to Trade} (with Van Dang)\\
	Conferences: Asian FA 2019, FMA 2019, FIRN 2019

	% {\bf Understanding Executive Compensation via Machine Learning} \\
	% Executive compensation is complex. This paper aims to provide empirical facts on executive compensation contracts using state-of-the-art machine learning models (trees and deep learning). I have two main findings. First, certain contract items have large effects on firm performance and many items seem redundant. Second, compensation mainly reflects efficient contracting than rent extraction: CEO's ability and talents explain much more variations in CEO compensation than governance variables, but rent extraction motives are also non-trivial.\\

	\begin{rSection}{Working Experience}
		\begin{rSubsection}{UNSW}{2022 \- }{Early Career Fellow (Level A)}{}
			\item Responsible for teaching several core postgraduate courses and contribute to the research of the School
		\end{rSubsection}
		\begin{rSubsection}{UNSW}{2020 - 2021}{PhD Teaching Fellow (Level A)}{}
			\item Responsible for teaching several core postgraduate courses
		\end{rSubsection}
		\begin{rSubsection}{Rozetta (formerly CMCRC)}{2017 - 2018}{Research Consultant / Business Analyst}{}
			\item Development of Research Platform (www.mqdashboard.com) to assist regulators and stock exchanges in assessing market quality and the effects of market design changes. Primary roles include developing algorithm and metrics of market quality assessment; Producing research reports to regulator and exchanges to assist their decision-making.
		\end{rSubsection}
	\end{rSection}

\end{rSection}
\begin{rSection}{Teaching Experience}
	\textbf{Lecturer-in-Charge}
	\begin{itemize}
		\item FINS5517 Applied Portfolio Management (2021)
		      \begin{itemize}
			      \item Teaching Evaluation: mean score 5.72 (Faculty 5.32)
		      \end{itemize}
		\item MFIN6201 Empirical Techniques and Application in Finance (2020, 2021, 2022)
		      \begin{itemize}
			      \item Most Recent Teaching Evaluation: mean score 5.73 (Faculty 5.31)
		      \end{itemize}
		\item MFIN6210 Empirical Studies in Finance (2020)
		      \begin{itemize}
			      \item Teaching Evaluation: mean score 5.58 (Faculty 5.22)
		      \end{itemize}
	\end{itemize}
\end{rSection}

\begin{rSection}{Professional Services}
	\begin{itemize}
		\item FMA International 2017 (discussant)
		\item CAFM 2017 (discussant)
		\item Asian FA 2019 (discussant)
		\item FIRN 2019 (discussant)
		\item FMCG 2022 (discussant)
	\end{itemize}
\end{rSection}

\begin{rSection}{Technical Skills}
	\begin{itemize}
		\item A Python developer, contributed to scientific packages such as pandas.
		\item Proficient in a variety of modern technologies: Apache Spark, SAS, Postgres, MongoDB, Stata, Machine Learning such as tree, deep neural networks and natural language processing.
	\end{itemize}
\end{rSection}

\begin{rSection}{Language}
	Chinese - Native\\
	English - Fluent
\end{rSection}

\begin{rSection}{Nationality}
	Australian Citizen
\end{rSection}

\begin{rSection}{Referees} \itemsep -3pt
	\begin{multicols}{2}
		Assoc. Prof. Wing Wah Tham\\
		University of New South Wales\\
		UNSW Business School\\
		Email: w.tham@unsw.edu.au

		Assoc. Prof. Elvira Sojli\\
		University of New South Wales\\
		UNSW Business School\\
		Email: e.sojli@unsw.edu.au\\
		\columnbreak

		Assoc. Prof. Rik Sen\\
		University of New South Wales\\
		UNSW Business School\\
		Email: rik.sen@unsw.edu.au
	\end{multicols}

\end{rSection}
\end{document}
